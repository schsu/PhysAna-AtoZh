\documentclass{article}

% packages
\usepackage{hyperref}
\usepackage{verbatim}

% document style
\usepackage[margin = 20mm]{geometry}
\setlength\parindent{0pt}

\begin{document}

\section*{Setting up MadGraph, Pythia, Delphes, and the 2HDMTC Model}

The 2HDMTC model is a model for decays involving the heavy Higgs particle, referred to as A or h3. This model currently does not work with the newest version of MadGraph, so there are some technical complications in getting the model to work with MadGraph, Pythia, and Delphes. This document aims to be a step-by-step manual for installing and configuring MadGraph, Pythia, Delphes, and the 2HDMTC model to create a working environment for heavy Higgs Montecarlo studies.

\section{Installing ROOT}

ROOT is a software package developed by CERN. It is used in most particle physics data analysis, and is required here to both generate and analyze Events. To install it, first download it from the following url using wget.

\begin{verbatim}
	wget ftp://root.cern.ch/root/root_v5.34.14.source.tar.gz
\end{verbatim}

Then untar it.

\begin{verbatim}
	tar -xzf root_v5.34.14.source.tar.gz
\end{verbatim}

Then cd to the untarred directory.

\begin{verbatim}
	cd root
\end{verbatim}

Then run ./configure.

\begin{verbatim}
	./configure
\end{verbatim}

And finally compile root with make.

\begin{verbatim}
	make
\end{verbatim}

Compiling root may take a very long time. After compiling root, you must run the script thisroot.sh using the command source.

\begin{verbatim}
	source root/bin/thisroot.sh
\end{verbatim}

\section{Installing MadGraph}

MadGraph is the software package which does our Montecarlo simulation. It takes process definitions and models as input, and outputs a process directory in which you modify simulation settings and generate Events. We will go over using MadGraph in detail later. For now, download it from the following url using wget.

\begin{verbatim}
	wget https://launchpad.net/mg5amcnlo/trunk/1.5.0/+download/MadGraph5_v1.5.14.tar.gz
\end{verbatim}

Then untar it.

\begin{verbatim}
	tar -xzf MadGraph5_v1.5.14.tar.gz
\end{verbatim}

\section {Installing Pythia}

Pythia generates particle showers from the partons generated by MadGraph. This will be explained in more detail later. For now download Pythia to the MadGraph directory from the following url using wget.

\begin{verbatim}
	wget http://madgraph.hep.uiuc.edu/Downloads/pythia-pgs_V2.2.0.tar.gz
\end{verbatim}

Then untar it.

\begin{verbatim}
	tar -xzf pythia-pgs_V2.2.0.tar.gz
\end{verbatim}

Then cd to the untarred directory.

\begin{verbatim}
	cd pythia-pgs
\end{verbatim}

Then cd to the src directory.

\begin{verbatim}
	cd src
\end{verbatim}

We need to edit the file "makefile" in this src directory. Find the following line in the makefile.

\begin{verbatim}
	Links = mass_width_2004.mc pgs clean_output PDFsets pydata.f
\end{verbatim}

And change it to the following.

\begin{verbatim}
	Links = mass_width_2004.mc PDFsets pydata.f
\end{verbatim}

Then cd back to pythia-pgs

\begin{verbatim}
	cd ..
\end{verbatim}

And finally compile pythia with make.

\begin{verbatim}
	make
\end{verbatim}

\section {Installing Delphes}

\begin{verbatim}
	cd MadGraph5_v1.5.14
\end{verbatim}

\begin{verbatim}
	wget http://cp3.irmp.ucl.ac.be/downloads/Delphes-3.0.12.tar.gz
\end{verbatim}

\begin{verbatim}
	tar -xzf Delphes-3.0.12.tar.gz
\end{verbatim}

\begin{verbatim}
	mv Delphes-3.0.12 Delphes
\end{verbatim}

\begin{verbatim}
	make
\end{verbatim}

\section {Installing 2HDMTC Model}

The 2HDMTC Model we use can be found in the git repository for this project. To get it, first clone the repository:

\begin{verbatim}
	git clone https://github.com/schsu/PhysAna-AtoZh
\end{verbatim}

\begin{verbatim}
	PhysAna-AtoZh/models/2HDMTC.tar
\end{verbatim}

\end{document}
