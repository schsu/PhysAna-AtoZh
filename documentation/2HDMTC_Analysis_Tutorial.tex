\documentclass{article}

% packages
\usepackage{hyperref}
\usepackage{verbatim}
\usepackage{amsmath}

% document style
\usepackage[margin = 20mm]{geometry}
\setlength\parindent{0pt}

\begin{document}

\section*{2HDMTC Analysis Tutorial}

This tutorial guides the user through various stages of data simulation and analysis of the hypothetical process $p p \rightarrow{} A$, $A \rightarrow{} Z h$, $h \rightarrow{} b \bar{b}$, $Z \rightarrow{} l+ l-$ (henceforth, AZH). The prerequisites to this tutorial are a basic understanding of collider physics, Linux, C++, and ROOT. Refer to the following links for an overview of these topics.

\bigskip

collider physics:

Introduction to Collider Physics - \url{http://arxiv.org/abs/1002.0274}

...

\bigskip

This tutorial is broken up into 4 sections - Level 1, Level 2, Level 3, and Level 4. In Level 1, you will install and configure the appropriate packages of ROOT, MadGraph, Pythia, Delphes, and the 2HDMTC Model in order to generate Montecarlo simulations of our AZH process. You will also learn about each step in the Montecarlo simulation process, that is, what exactly MadGraph, Pythia, and Delphes are doing, as well as how to edit various configuration files to modify your simulation. Then, in Level 2, you will learn how to analyze the truth particle data in the ROOT files generated by your simulation. In Level 3, you will learn how to perform more extensive analysis on the detected particle data in your ROOT files, both before and after various cuts. Finally, in Level 4, you will learn how to manually analyze jets, with the motivation of creating a good jet reconstruction algorithm.

\section{Level 1: Generating Simulated Events with MadGraph, Pythia, Delphes, and 2HDMTC}

The 2HDMTC model is a model for decays involving the heavy Higgs particle, referred to as A or h3. This model currently does not work with the newest version of MadGraph, so there are some technical complications in getting the model to work with MadGraph, Pythia, and Delphes. This section aims to be a step-by-step manual for installing and configuring MadGraph, Pythia, Delphes, and the 2HDMTC model to create a working environment for heavy Higgs Montecarlo studies. The details of the Montecarlo simulation are then discussed, and finally editing the configuration of the simulation is explained.

\subsection{Installing ROOT}

ROOT is a software package developed by CERN. It is used in most particle physics data analysis, and is required here to both generate and analyze Events. To install it, first download it from the following url using wget.

\begin{verbatim}
	wget ftp://root.cern.ch/root/root_v5.34.14.source.tar.gz
\end{verbatim}

Then untar it.

\begin{verbatim}
	tar -xzf root_v5.34.14.source.tar.gz
\end{verbatim}

Then cd to the untarred directory.

\begin{verbatim}
	cd root
\end{verbatim}

Then run ./configure.

\begin{verbatim}
	./configure
\end{verbatim}

And finally compile root with make.

\begin{verbatim}
	make
\end{verbatim}

Compiling root may take a very long time. After compiling root, you must run the script thisroot.sh using the command source.

\begin{verbatim}
	source root/bin/thisroot.sh
\end{verbatim}

\subsection{Installing MadGraph}

MadGraph is the software package which does our Montecarlo simulation. It takes process definitions and models as input, and outputs a process directory in which you modify simulation settings and generate Events. We will go over using MadGraph in detail later. For now, download it from the following url using wget.

\begin{verbatim}
	wget https://launchpad.net/mg5amcnlo/trunk/1.5.0/+download/MadGraph5_v1.5.14.tar.gz
\end{verbatim}

Then untar it.

\begin{verbatim}
	tar -xzf MadGraph5_v1.5.14.tar.gz
\end{verbatim}

\subsection {Installing Pythia}

Pythia generates particle showers from the partons generated by MadGraph. This will be explained in more detail later. For now download Pythia to the MadGraph directory from the following url using wget.

\begin{verbatim}
	wget http://madgraph.hep.uiuc.edu/Downloads/pythia-pgs_V2.2.0.tar.gz
\end{verbatim}

Then untar it.

\begin{verbatim}
	tar -xzf pythia-pgs_V2.2.0.tar.gz
\end{verbatim}

Then cd to the untarred directory.

\begin{verbatim}
	cd pythia-pgs
\end{verbatim}

Then cd to the src directory.

\begin{verbatim}
	cd src
\end{verbatim}

We need to edit the file "makefile" in this src directory. Find the following line in the makefile.

\begin{verbatim}
	Links = mass_width_2004.mc pgs clean_output PDFsets pydata.f
\end{verbatim}

And change it to the following.

\begin{verbatim}
	Links = mass_width_2004.mc PDFsets pydata.f
\end{verbatim}

Then cd back to pythia-pgs

\begin{verbatim}
	cd ..
\end{verbatim}

And finally compile pythia with make.

\begin{verbatim}
	make
\end{verbatim}

\subsection {Installing Delphes}

Delphes simulates the detector response to particle physics processes. It is explained in more detail later. For now, download Delphes to the MadGraph directory using the following url with wget.

\begin{verbatim}
	wget http://cp3.irmp.ucl.ac.be/downloads/Delphes-3.0.12.tar.gz
\end{verbatim}

Then untar it.

\begin{verbatim}
	tar -xzf Delphes-3.0.12.tar.gz
\end{verbatim}

Now change the untarred directory's name to "Delphes".

\begin{verbatim}
	mv Delphes-3.0.12 Delphes
\end{verbatim}

Finally, cd to Delphes and run "make".

\subsection {Installing 2HDMTC Model}

The 2HDMTC Model we use can be found in the git repository for this project. To get it, first clone the repository:

\begin{verbatim}
	git clone https://github.com/schsu/PhysAna-AtoZh
\end{verbatim}

Then locate the model from the following directory in the repository:

\begin{verbatim}
	PhysAna-AtoZh/models/2HDMTC.tar
\end{verbatim}

And copy it into your MadGraph directory. Finally, untar the model:

\begin{verbatim}
	tar -xzf 2HDMTC.tar
\end{verbatim}

\subsection {MadGraph Explanation}

\subsection {Pythia Explanation}

\subsection {Delphes Explanation}

\subsection {Modifying Configuration Files}

\section{Level 2: Truth Particle Study}

\end{document}
